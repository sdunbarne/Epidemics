%%% -*-LaTeX-*-
%%% epidemics.tex.orig
%%% Prettyprinted by texpretty lex version 0.02 [21-May-2001]
%%% on Fri Jul  1 08:32:07 2022
%%% for Steve Dunbar (sdunbar@family-desktop)

\documentclass[12pt]{article}

\input{../../../../etc/macros} % \input{../../../../etc/mzlatex_macros}
\input{../../../../etc/pdf_macros}

\bibliographystyle{plain}

\begin{document}

\myheader \mytitle

\hr

\sectiontitle{Markov Chain Epidemic Models}

\hr

\visual{Rating}{../../../../CommonInformation/Lessons/rating.png}
\section*{Rating} %one of
% Everyone: contains no mathematics.
% Student: contains scenes of mild algebra or calculus that may require guidance.
Mathematically Mature:  may contain mathematics beyond calculus with
proofs.  % Mathematicians Only: prolonged scenes of intense rigor.

\hr

\visual{Section Starter Question}{../../../../CommonInformation/Lessons/question_mark.png}
\section*{Section Starter Question}

Suppose a single person in a population of \( S \) susceptible
individuals has a communicable disease.  What is a simple probabilistic
model of how this person can infect some number of susceptible
individuals in a given time interval?

\hr

\visual{Key Concepts}{../../../../CommonInformation/Lessons/keyconcepts.png}
\section*{Key Concepts}

\begin{enumerate}
    \item
        Markov chain \defn{Susceptible-Infected-Removed (SIR) models}.
        have a population and a disease which infects members of the
        population for a fixed amount of time, after which the infected
        individuals are removed, in the sense that each one is no longer
        susceptible and cannot become infected again.
    \item
        The \defn{Greenwood Model} assumes the number of Infectives in
        generation \( t \) is a binomial random variable with parameters
        \( S_t \) and infection success constant \( p \), not depending
        on the number of Infectives.
    \item
        For the Greenwood model
        \begin{align*}
            \E{S_t \given S_0= s_0} &= (1-p)^t s_0, \\
            \E{I_t \given S_0= s_0} &= p(1-p)^{t-1} s_{0}.
        \end{align*}
    \item
        The \defn{Reed-Frost model} includes an infective size
        dependency that the Greenwood model lacks.
    \item
        In comparison to the Greenwood model, the Reed-Frost model gives
        a higher probability to more new Infectives at generation \( t+1
        \) if there are already a large number of Infectives at
        generation \( t \) than if there is only a small population of
        Infectives. The Reed-Frost model has a positive feedback effect
        on infectives.
    \item
        Under some common asymptotic assumptions, is, the number of
        Infectives is the result of a Galton-Watson process or branching
        process.  Then from the standard theory for branching processes,
        the epidemic is \emph{subcritical} if \( \lambda \le 1 \) and
        \emph{supercritical} if \( \lambda > 1 \).
\end{enumerate}

\hr

\visual{Vocabulary}{../../../../CommonInformation/Lessons/vocabulary.png}
\section*{Vocabulary}
\begin{enumerate}
    \item
        Markov chain \defn{Susceptible-Infected-Removed (SIR) models}.
        have a population and a disease which infects members of the
        population for a fixed amount of time, after which the infected
        individuals are removed, in the sense that each one is no longer
        susceptible and cannot become infected again.
    \item
        The \defn{Greenwood Model} assumes the number of Infectives in
        generation \( t \) is a binomial random variable with parameters
        \( S_t \) and infection success constant \( p \), not depending
        on the number of Infectives.
    \item
        The \defn{Reed-Frost model} includes an infective size
        dependency that the Greenwood model lacks.
    \item
        The \defn{extinction time \( \tau \) of an epidemic} to be the
        generation at which \( i_t \) is first zero.
    \item
        The \defn{total damage} of the epidemic is \( K=\sum_{j = 0}^{\tau}
        i_j \).
\end{enumerate}

\hr

\section*{Notation}
\begin{enumerate}
    \item
        \( t \) -- the integer variable enumerating unit time periods
        for each generation
    \item
        \( N \) -- population constant value
    \item
        \( S_t \) -- the number of Susceptibles at generation \( t \)
    \item
        \( I_t \) -- the number of Infectives at generation \( t \)
    \item
        \( p \) -- the probability that a given Susceptible will become
        infected in one generation
    \item
        \( K=\sum_{j = 0}^{\tau} I_j \) -- the total damage of the
        epidemic
    \item
        \( q = 1 - p \) -- the probability of \emph{avoiding} contact
        with that infective
    \item
        \( \tau \) -- extinction time of an epidemic to be the
        generation at which \( i_t \) is first zero
    \item
        \( s_t \) -- specific numbers of Susceptibles.
    \item
        \( i_{t} \) -- the specific number of Susceptibles who will
        become infected
    \item
        \( m = i_0 = o(n) \)
    \item
        \( \lambda \) -- parameter such that \( q = \EulerE^{-\lambda/n}
        \)
\end{enumerate}
\visual{Mathematical Ideas}{../../../../CommonInformation/Lessons/mathematicalideas.png}
\section*{Mathematical Ideas}

\subsection*{Introduction}

This section is an introduction to elementary Markov chain \defn{Susceptible-Infected-Removed
(SIR) models}.~%
\index{SIR models}%
~%
\index{epidemic models}
These Markov chains model a population and a disease that infects
members of the population for a fixed amount of time, after which each
infected individual are removed, in the sense that they are no longer
susceptible and cannot become infected again.  The removed class
includes individuals who are no longer either susceptible either because
of recovery and subsequent immunity or because of death.  That is, after
members of the population have been infected for one time unit, the
model no longer considers them, as the removed indiduals cannot infect
anyone else, nor can they become infected again.  Therefore only two
classes need to be considered, the Susceptibles and the Infectives.
Figure~%
\ref{fig:epidemics:sirdiagram} shows the movement between the classes.

\begin{figure}
    \centering
\begin{asy}
        settings.outformat = "pdf";

        // import graph;

        size(5inches);

        real myfontsize = 12;
        real mylineskip = 1.2*myfontsize;
        pen mypen = fontsize(myfontsize, mylineskip);
        defaultpen(mypen);

        real r = 0.10;

        path cS = circle( (0,0), r);
        pair cSa = relpoint(cS, 0);
        draw(cS);
        label("\( S \)", (0,0));

        path cI = circle( (1/2,0), r);
        pair cIa = relpoint(cI, 0);
        pair cIz = relpoint(cI, 0.5);
        draw(cI); label("\( I \)", (1/2, 0));

        path cR = circle( (1,0), r);
        pair cRz = relpoint(cR, 0.5);
        draw(cR);
        label("\( R \)", (1, 0));

        draw( cSa -- cIz, Arrow);
        draw( cIa -- cRz, Arrow);
\end{asy}
    \caption{Schematic class diagram for the simple
    Susceptible-Infective-Removed epidemic model illustrating the
    classes of the epidemic and the movement from class to class and not
    the state transitions of the Markov chain.}%
    \label{fig:epidemics:sirdiagram}
\end{figure}

Markov chain models use discrete time and a discrete state space.  The
discrete time will be in units of the infectivity period of the disease.
Typically this unit time is about \( 7 \)-\( 8 \) days for measles, and
about \( 7 \)-\( 9 \) days for influenza.  Call each of these unit time
periods a generation, enumerated by the integer variable \( t \).  (This
is a slight change from the usual notation for discrete time Markov
chains using indices \( i \) or \( j \) for the discrete time.  Using \(
t \) avoids confusion with the number of Infectives.) The infectivity
period may either be distinct from, or overlap, the period when the
infected person shows symptoms of the disease.  The incubation period,
the period between infection and onset of symptoms, and the latent
period, the time from infection to infectiousness are different from the
infectivity period.  Figure~%
\ref{fig:epidemics:periods} illustrates one possible configuration of
these disease periods.  The states will be the number of Susceptibles
and Infectives in a homogeneous and well-mixed population.  Assume the
epidemic takes place over a short enough time interval that the
population is a constant value \( N \), that is, assume no births and no
emigration or immigration.  So it is enough to consider only
Susceptibles and Infectives since the number of individuals in the
Removed class is determined from the other two.

\begin{figure}
    \centering
\begin{asy}
        settings.outformat = "pdf";

        import graph;

        size(5inches);

        real myfontsize = 12;
        real mylineskip = 1.2*myfontsize;
        pen mypen = fontsize(myfontsize, mylineskip);
        defaultpen(mypen);

        real f( real x) { return -(x - 2/3) * (x - 4/3);}

        path sympt = graph(f, 2/3, 4/3);
        draw( sympt ); fill(sympt--(4/3,0)--cycle, lightgray);

        real[] rt = {1/2, 1.0}; real[] lt = {2/3, 4/3};

        xaxis(L=Label("Time", position=EndPoint), xmin=0, xmax=3/2,
              ticks=LeftTicks(lt, format="%"), arrow=Arrow);
        xaxis(xmin=0, xmax=1.2, ticks=RightTicks(rt, format="%"));
        yaxis(L=Label("Symptoms", position=EndPoint), ymin=0, ymax=0.3,
              arrow=Arrow);

        label("Latent Period", (1/4,0), S);
        label("Infectious Period", (3/4,0), S);
        label("Incubation Period", (1/3,0), N);
        label("Symptoms Present", (1,0), N);
\end{asy}
    \caption{Possible relation of disease stages or periods.}%
    \label{fig:epidemics:periods}
\end{figure}

To keep track of each group, denote the number of Susceptibles at
generation \( t \) by \( S_t \), and the number of Infectives at
generation \( t \) by \( I_t \).  These random variables may achieve any
integer value such that \( S_t + I_t \) is in the population range.

A natural first question for any epidemic model is the probability of
having \( I_{t+1} \) Infectives at generation \( t+1 \).  In asking this
question at generation \( t \), the number of Susceptibles and
Infectives at every generation up to \( t \) are known.  Since the model
is expressed in time units of infectivity periods where individuals move
from one class to another, the main focus is on the number of Infectives
at generation \( t \), ignoring the past, also keeping track of the
Susceptibles.  Then the model is a first-order Markov process through
the states \( S_t \), \( I_t \) with \( S_t + I_t \le N \).  If the
population is \( N \), the number of states is \( N(N+1)/2 \) so this is
a Markov chain with many states.

\subsection*{Greenwood Model}

The \defn{Greenwood Model}~%
\index{Greenwood model}
assumes the number of Infectives in generation \( t \) is a binomial
random variable with parameters \( S_t \) and infection success constant
\( p \), not depending on the number of Infectives:
\begin{multline*}
    \Prob{I_{t+1}=i_{t+1} \given S_0=s_0, I_0=i_0, \ldots,S_t=s_t, I_t=i_t}
    = \\
    \Prob{I_{t+1}=i_{t+1} \given S_t=s_t, I_t=i_t} = \binom{s_t}{i_{t+1}}p^
    {i_{t+1}}(1-p)^{s_t-i_t}.
\end{multline*}

The first equality represents the first-order Markov chain assumption.
Consider the second equality in further detail.  At generation \( t \)
there are \( s_t \) Susceptibles.  To find the probability that \( i_{t+1}
\) members of this group become Infectives, choose \( i_{t} \) of the
Susceptibles who will become infected.  Similarly \( p \) is the
probability that a given Susceptible will become infected in one
generation.  Thus the probability \( i_{t+1} \) Susceptibles become
Infectives is \( p^{i_{t+1}} \).  Note that a Susceptible becoming
Infective is a ``success'' (for the disease, not the population) in the
binomial random variable sense.  Remaining a Susceptible is a failure in
the binomial sense.  The transition for the Susceptibles is
\[
    S_{t+1} = S_t - I_{t+1}.
\]

Because \( I_{t+1} \) is a binomial random variable
\[
    E[I_{t+1} \given S_t]= p S_t.
\] This expression gives us the expected number of Infectives at
generation \( t+1 \) given the number of Susceptibles at generation \( t
\).  Extend this idea to say
\begin{align*}
    \E{S_{t+1} \given S_t } = \E{S_t} - \E{I_{t+1}} = (1-p) S_t.
\end{align*}
By induction
\begin{align*}
    \E{S_t \given S_0= s_0} &= (1-p)^t s_0, \\
    \E{I_t \given S_0= s_0} &= p(1-p)^{t-1} s_{0}.
\end{align*}

\subsection*{Reed-Frost Model}

The Reed-Frost model is an SIR mathematical model of epidemics created
in the 1920s by Lowell Reed and Wade Hampton Frost at Johns Hopkins
University.  Originally presented in a talk by Frost in 1928 and used in
courses at Hopkins for two decades, the mathematical formulation was not
published until the 1950s.

The Greenwood model has a fixed probability of infection, independent of
the number of Infectives at the current generation step.  The \defn{Reed-Frost
model}~%
\index{Reed-Frost model}
includes an infective size dependency that the Greenwood model lacks.
The Reed-Frost model is based on the following assumptions:
\begin{enumerate}
    \item
        The infection is spread directly from Infectives to Susceptibles
        by a certain type of contact (termed ``adequate contact'') and
        in no other way.
    \item
        Any Susceptible in the group, after such contact with an
        Infective in a given generation, will develop the infection and
        will be infectious to others only within the following
        generation period.  In subsequent generation periods, the
        individual is wholly and permanently immune and moves to the
        Removed class.
    \item
        Each individual has a fixed probability \( p \) of coming into
        adequate contact with any other specified individual in the
        group within one time interval, and this probability is the same
        for every member of the group.
    \item
        The population is constant with no births, immigration or
        emigration.
    \item
        These conditions stay constant during the epidemic.
\end{enumerate}
Set the following initial parameters:
\begin{itemize}
    \item
        the size of the population \( N \),
    \item
        the number of individuals already immune, typically \( 0 \),
    \item
        the number of Infective cases, usually set at \( 1 \),
    \item
        the probability \( p \) of adequate contact.
\end{itemize}
If \( p \) is the probability of a given Susceptible \emph{making
adequate contact} with \( 1 \) specific Infective, then \( q = 1 - p \)
is the probability of \emph{avoiding} contact with that infective.  Make
the (vastly simplifying and probably unjustified) assumption that all
interactions of the Susceptible with all Infectives are independent
events.  Then \( q^{i_t} \) is the probability the given Susceptible at
generation \( t \) avoids adequate contact at generation \( t \) from
\emph{all} Infectives in the population at that generation and remains
Susceptible at generation \( t + 1 \).  This is a failure for the
epidemic in the binomial random variable sense.  The probability of a
Susceptible being infected is then \( (1 - q^{i_t}) \), a success for
the epidemic.  Then as a binomial random variable, the probability of
the number of Infectives in the next generation is
\[
    \Prob{I_{t+1} = i_{t+1} \given S_t=s_t, I_t=i_t} = \binom{s_t}{i_{t+1}}
    \left(1 - (1-q^{i_t})\right)^{ i_{t+1}}\left(q^{i_t}\right)^{s_t-i_t}
\] and \( S_{t+1} = S_t - I_{t+1} \).  In comparison to the Greenwood
model, the Reed-Frost model gives a higher probability to more new
Infectives at generation \( t+1 \) if the population already has a large
number Infectives at generation \( t \) than if there is only a small
population of Infectives.  The Reed-Frost model has a positive feedback
effect on infectives.

A success for the epidemic is a failure for the Susceptibles and a
failure for the epidemic is a success for the Susceptibles.  Then an
alternative expression for the Reed-Frost model is
\[
    \Prob{S_{t+1} = s_{t+1} \given S_t=s_t, I_t=i_t} = \binom{s_t}{i_{t+1}}
    \left( q^{i_t}\right)^{i_{t+1}}\left(1 - q^{i_t}\right)^{s_t-i_t}
\] and \( S_{t+1} = S_t - S_{t+1} \).

Define the \defn{extinction time \( \tau \) of an epidemic}%
\index{extinction time}
to be the generation at which \( i_t \) is first zero.  This is an
absorbing state for the Markov Chain.  Note that if the number of
Susceptibles becomes \( 0 \) in a generation \( t \), then \( i_{t+1} =
0 \).  Of more interest is the probability \( S_{\tau} > 0 \) and \( I_{\tau}
= 0 \).  Consider an epidemic for which the Infectives have progressed
to extinction \( i_1, i_2, \ldots, i_{\tau-1}, i_{\tau} = 0 \).  Note
that \( i_{\tau - 1} > 0 \).  Thus
\begin{multline*}
    \Prob{I_1 =i_1, \ldots, I_{k}=i_k, I_{k+1}=0 \given S_0=n, I_0=m} =
    \\
    \Prob{I_1=i_1 \given S_0=n, I_0=m}\times \Prob{I_2=i_2 \given S_1=n-i_1,
    I_1=i_1}\times\cdots\times \\
    \Prob{I_{\tau}=0 \given S_{\tau - 1}=s_{\tau - 1}, I_{\tau - 1}=i_{\tau
    - 1}}.
\end{multline*}

In this specific epidemic, the \defn{total damage}%
\index{total damage}
of the epidemic is \( K=\sum_{j = 0}^{\tau} I_j \).  As \( K \) is the
sum of \( I_i \)'s, treat it as a random variable.  Thus consider:
\[
    \Prob{K=k \given S_0=n,I_0=m}=\sum_{\vec{i}:  \abs{\vec{i}=k}}\Prob {I_1
    =i_1,\ldots,I_{\tau}=0 \given S_0=n,I_0=m}.
\] Some common asymptotics apply here.  Let \( m = i_0 = o(n) \) and
suppose \( q = \EulerE^{-\lambda/n} \) so \( p = 1- q = O(n^{-1}) \) and
the expected number of adequate contacts \( q^{i_t} \) is approximately
constant.  At the beginning of the epidemic, using the first formulation
of the Reed-Frost model
\begin{align*}
    \Prob{I_{t+1}} &=
    \operatorname{Bin}
    (S_t, 1 - \EulerE^{-\lambda I_t/n}) \\
    &\approx
    \operatorname{Bin}
    (n, \lambda I_t/n) \\
    &\approx
    \operatorname{Poisson}
    (\lambda I_t)
\end{align*}
This means \( I_{t+1} \) is the sum of \( I_t \) independent Poisson
random variables with parameter \( \lambda \).  That is, the number of
Infectives is the result of a Galton-Watson process or branching
process.  Then from the standard theory for branching processes, the
epidemic is \emph{subcritical} if \( \lambda \le 1 \) and \emph{supercritical}
if \( \lambda > 1 \).  That is, the trajectories of this process either
die out or explode in an exponential way, at least at the beginning of
the epidemic while the asymptotic assumptions hold.  This points to more
general stochastic process models of epidemics as branching processes.
Although the epidemic models are Markov chains, the tools of branching
processes are more applicable and yield more detailed results.

\subsection*{Limitations}

Both of these models assume a homogeneous population, i.e.\ everyone is
equally likely to get sick assuming the same exposure.  Also the
Reed-Frost model assumes that the population mixes perfectly, that is,
every Susceptible interacts with every Infective and thus has an equally
likely chance to get sick from each one.  Both models assume
independence in these interactions.  These assumptions apply best to a
moderately-sized (reasonably) closed situations such as a dormitory, a
nursing home under quarantine, a cruise ship, a factory, or perhaps a
small town.

For large populations such as a city or an entire country, the
assumptions of homogeneous population and perfect mixing don't apply.
Also for large populations, direct calculation of the binomial
coefficients at each stage is inconvenient, and so using continuum
limits is appropriate, leading to differential equation epidemic models.
However, one advantage of the stochastic process models is the
possibility of obtaining a distribution for the extinction times.  This
is a more realistic property than the extinction time given by an
asymptotic deterministic model.

More detailed models have multiple health and infection classes instead
of just Infectives.  More detailed models may also cross these with
factors influencing mixing, such as age, social class, or geographical
location.  The mathematical modeling of epidemics has a huge literature,
the references below are a starting point.

\visual{Section Starter Question}{../../../../CommonInformation/Lessons/question_mark.png}
\section*{Section Ending Answer}

One particularly simple model is that the infective transmits the
disease to a fixed number of susceptibles, this is not probabilistic.  A
probabilistic model is that the infective transmits to each susceptible
with probability \( p \), leading to the binomial probability
distribution of the Greenwood model.  A third possibility is that the
number of infected is a Poisson random variable with parameter \(
\lambda \).  This would be appropriate if the population is large and
the transmission rate is constant across various size groups of
susceptibles.

\subsection*{Sources}

Details of the Greenwood and Reed-Frost models are adapted from
``Criticality in Epidemic Models'' by R. Dolgoarshinnykh,
\cite{dolgoarshinnykh}.  Comments on epidemics as branching processes
and more detailed assumptions for epidemics are from ``Branching
Processes:  Their Role in Epidemiology'', by C. Jacob,
\cite{jacob10}.  See also
\cite{siettos13} for a survey of mathematical models of infectious
diseases.  The exercise is adapted from the Ohio Supercomputer Center
Summer Institute, \link{https://www.osc.edu/education/si/projects/epidemic}
{Reed-Frost Epidemic Model}.

\hr

\visual{Algorithms, Scripts, Simulations}{../../../../CommonInformation/Lessons/computer.png}
\section*{Algorithms, Scripts, Simulations}

\subsection*{Algorithm}

\begin{algorithm}[H]
  \DontPrintSemicolon
  \KwData{initial Susceptibles, Infectives, probability, time}
  \KwResult{Plot of Susceptibles, Infectives versus time}
  \BlankLine
  \emph{Initialization}\;
  Set initial Susceptibles, Infectives, probability, time\;
  Loop over time\;
  Get random binomial variate from number of Susceptible\;
  Update the infectives\;
  \KwRet{Plot of Susceptibles, Infectives versus time}
  \caption{Simulation of epidemics with Greenwood and Reed Frost models.}
\end{algorithm}

\subsection*{Scripts}


\begin{description}

% \item[Geogebra] 

% \link{  .ggb}{GeoGebra applet}

\item[R] 

\link{greenwood.R}{R script for Greenwood model.}

\begin{lstlisting}[language=R]
S0 <- 999
I0 <- 1
p  <- 0.2

T <- 30
S <- numeric(T + 1)
I <- numeric(T + 1)

S[1]  <- S0
I[1]  <- I0
for (t in 2:(T + 1)) {
    I[t]  <- rbinom(1, size=S[t-1], prob=p)
    S[t] <-  S[t - 1] - I[t]
}

plot(1:(T + 1), S, col="blue",
     xlim = c(0, (T + 1)), ylim=c(0, (S0 + I0)),
     main="Greenwood Epidemic Model",
     xlab="Time", ylab="S, I")
points(1:(T + 1), y=I, col="red")
\end{lstlisting}

\link{reedfrost.R}{R script for Reed-Frost model.}

\begin{lstlisting}[language=R]
S0 <- 9999
I0 <- 1
p  <- 0.00001
q  <- 1 - p

T <- 25
S <- numeric(T + 1)
I <- numeric(T + 1)

S[1]  <- S0
I[1]  <- I0
for (t in 2:(T + 1)) {
    QT = q^I[t-1]
    I[t]  <- rbinom(1, size=S[t-1], prob=1-QT)
    S[t] <-  S[t - 1] - I[t]
}

plot(1:(T + 1), S, col="blue",
     xlim = c(0, (T + 1)), ylim=c(0, (S0 + I0)),
     main="Reed-Frost Epidemic Model",
     xlab="Time", ylab="S, I")
points(1:(T + 1), y=I, col="red")
\end{lstlisting}

% \item[Octave]

% \link{http://www.math.unl.edu/~sdunbar1/    .m}{Octave script for .}

% \begin{lstlisting}[language=Octave]

% \end{lstlisting}

% \item[Perl] 

% \link{http://www.math.unl.edu/~sdunbar1/    .pl}{Perl PDL script for .}

% \begin{lstlisting}[language=Perl]

% \end{lstlisting}

% \item[SciPy] 

% \link{http://www.math.unl.edu/~sdunbar1/    .py}{Scientific Python script for .}

% \begin{lstlisting}[language=Python]

% \end{lstlisting}

\end{description}



%%% Local Variables:
%%% TeX-master: t
%%% End:


\hr

\visual{Problems to Work}{../../../../CommonInformation/Lessons/solveproblems.png}
\section*{Problems to Work for Understanding}
\renewcommand{\theexerciseseries}{}
\renewcommand{\theexercise}{\arabic{exercise}}

\begin{exercise}
    Write a script that simulates a place-based epidemic model.  For
    this modification, assume that all the individuals are standing next
    to each other on a two-dimensional grid.  Let one individual be an
    infective case.  An infective can now probabilistically infect only
    his immediate neighbors on the grid.  In the next time period, those
    people who got sick now have a chance to infect their neighbors and
    so on.  Display on a two-dimensional grid that shows all the
    individuals what state they are in (i.e., Susceptible, Infective, or
    Recovered), with an animation if possible.  Run the script many
    times with different input parameters.  How does the pattern of the
    epidemic flow change with the contact probability?
\end{exercise}

\begin{solution}
    The following script is a simple implementation of an epidemic on a
    two-dimensional grid.  For a given value of \( N \), the script
    defines a population of Susceptibles on a square grid of size \( 2N
    + 1 \), and sets an Infective at the center location \( (N, N) \).
    The infective can infect neighbors with probability \( p \), ands
    fails to infect neighbors with probability \( 1 - p \).  Because the
    infection can only spread to immmediate neighbors, the epidemic
    spreads through concentric rings of squares outward from the center.
    The spread in concentric rings means that the epidemic will take
    place over \( N \) steps, possibly stopping at some earlier step due
    to no Infectives, which the script reports.  At each step, for all
    Infectives in the grid, the script successively infects immediate
    neighbors to the north, east, west and south with probability \( p \).
    After possibly infecting neighbors, the infective becomes a Removed
    individual, so the infectious period lasts one step.  At the end of
    each step, the script records the state of the epidemic.  For
    testing and debugging, a value of \( N = 3 \) is convenient.  For an
    experiment, a value of \( N = 25 \) is useful.

\begin{lstlisting}{language=R}
N <- 25
# N <- 3
p <- 0.6

popn <- matrix("S", 2*N+1, 2*N+1)
popn[N+1,N+1] <- "I"

dir <- list("N" = -1, "E" = 2*N+1, "S" = 1, "W" = -(2*N+1))

infect <- function(pos, d) {
       if (popn[pos + d] == "S") {
           if (runif(1) < p ){ popn[pos + d] <<- "I"}
       }
}

epidemic <- array("", dim=c(2*N+1, 2*N+1, N)) 

for (t in 1:N) { # RINGS or EPOCHS
    
    infs <- which(popn == "I")
    if (length(infs) == 0) {
        cat("Epidemic ended after", t-1, "steps.\n")
         break}
    else  { 
     for  (j in infs) {
          for (d in dir) {
              infect(j, d)
          }
     popn[j] <- "R"
     }
    epidemic[, , t] <- popn
   }
}   

# # Animation code is adapted from
# # https://ryouready.wordpress.com/2010/11/21/animate-gif-images-in-r-imagemagick/
# epidemicfactors <- array(as.numeric(factor(epidemic)), dim=c(2*N+1, 2*N+1, N))
# par(xaxt="n", yaxt="n")         #remove axis labels and ticks
# png(file="epidemicImage%02d.png")
# for (t in 1:N) {
#     image(epidemicfactors[ , , t], col=c( "red", "blue", "green" ))
# }
# dev.off()
# system("convert -delay 80 *.png epidemicAnimation.gif")

# file.remove(list.files(pattern=".png"))
\end{lstlisting}

    This script is a very simple epidemic model with only basic attempts
    at efficiency.  Many modifications are possible.

    The animation is simple, relying on the graphics manipulation tool
    ImageMagick.  The animation code is commented out because of the
    reliance on system dependent features.  More elaborate animations
    are possible with other R packages such as plotly and gganimate.

\end{solution}
\hr

\visual{Books}{../../../../CommonInformation/Lessons/books.png}
\section*{Reading Suggestion:}

\bibliography{../../../../CommonInformation/bibliography}

\hr

\section*{Outside Readings and Links:}
\begin{enumerate}
    \item
        \booktitle{Introduction to Probability}, Grinstead and Snell.
        Chapter 11 - Markov Chains - \\
        http://tinyurl.com/qw6sa
    \item
        \link{https://www.ncbi.nlm.nih.gov/pmc/articles/PMC3182455/}{Generalized
        Markov Models of Infectious Disease Spread}
    \item
        \link{http://www.stat.columbia.edu/~regina/research/risk.pdf}{Criticality
        in Epidemic Models}
\end{enumerate}

\section*{\solutionsname}
\loadSolutions

\hr

\mydisclaim \myfooter

Last modified:  \flastmod

\end{document}

